\chapter{Heterogeneous Cooperative Spectrum Sensing Code}
\section{harddecisionpdroc.m}
\begin{lstlisting}[breaklines]
% Hard-Decision Combining Results For Sensor Nodes
clc;
close all;
clear all;
%% Parameter Initialization
N = 32;
k=4;%sensor nodes..
variance = 24.32e-9;
pfa = 0.05;
threshold = (qfuncinv(pfa)+sqrt(N)).*sqrt(N)*2*variance;
snrthreotical = -18:0.5:20;
snrlinear = 10.^(snrthreotical/10);
%% SNR values from USRP and RTL-SDR
snrpracticalavg = [-18.23,-14.22,-13.1,-12.22,-10.35,-8.25,-5.3,-2.1,0.13,...
    1.34,2.58,3.76,8.98,11.33,12.35,13.45,15.77];
snrlinearprac = 10.^(snrpracticalavg/10);
%% Computing Detection Probability and ROC Characteristics
pdprac = qfunc((threshold-2*N*variance.*(1+snrlinearprac))./...
    (sqrt(N.*(1+2*snrlinearprac))*(2*variance)));
pdpracor = 1-(1-pdprac).^4;
pdpracand = pdprac.^4;
tmp1 =  (1-pdprac).^2;
tmp2 = (1-pdprac);
pdpracmjr = (6*pdprac.^2).*tmp1+(4*pdprac.^3).*tmp2+pdprac.^4;

pfapracor = 1-(1-pfa).^4;
pfapracand = pfa.^4;
tmp1 =  (1-pfa).^2;
tmp2 = (1-pfa);
pfapracmjr = (6*pfa.^2).*tmp1+(4*pfa.^3).*tmp2+pfa.^4;

%% ROC Characteristics...
figure(1)
hold on;
grid on;
plot(pfapracor,pdpracor(:,5),'-->','LineWidth',2,'MarkerFaceColor','auto');
plot(pfapracor,pdpracor(:,7),'-->','LineWidth',2,'MarkerFaceColor','auto');
plot(pfapracand,pdpracand(:,5),'-d','LineWidth',2,'MarkerFaceColor','auto');
plot(pfapracand,pdpracand(:,7),'-d','LineWidth',2,'MarkerFaceColor','auto');
plot(pfapracmjr,pdpracmjr(:,5),'--','LineWidth',2,'MarkerFaceColor','auto');
plot(pfapracmjr,pdpracmjr(:,7),'--','LineWidth',2,'MarkerFaceColor','auto');
xlabel('Average Probability Of False Alarm');
ylabel('Average Probability of Detection');
title('ROC Characterisitcs of Hard Decision Combining');
hold off;
set(gca,'fontsize',30,'box','on','LineWidth',2,'GridLineStyle','--','GridAlpha',0.7);
lgd = legend('OR SNR=-10.35dB','OR SNR=-5.3dB','AND SNR=-10.35dB',...
    'AND SNR=-5.3dB','Majority SNR=-10.35dB','Majority SNR=-5.3dB');
lgd.FontSize=20;

%% Probability of detection
figure(2)
hold on;
grid on;
plot(snrpracticalavg,pdpracor,'->','LineWidth',2);
plot(snrpracticalavg,pdpracand,'-<','LineWidth',2);
plot(snrpracticalavg,pdpracmjr,'-d','LineWidth',2);
xlabel('SNR_{avg} in (dB)');
ylabel('Average Probability of Detection');
title('P_{davg} vs SNR_{avg} for Hard Decision Combining');
hold off;
set(gca,'fontsize',30,'box','on','LineWidth',2,'GridLineStyle','--','GridAlpha',0.7);
lgd = legend('OR Decision','AND Decision','Majority Rule Decision');
lgd.FontSize=20;
axis([-18.23 15.77 0 1])
\end{lstlisting}


\section{softharddecisionpd.m}
\begin{lstlisting}[breaklines]
% Soft Decision Combining for sensor nodes...
%% Initializing parameters..
close all;
clear all;
N = [100,200,300,400];% Different sum factor
k=4;% Number of Sensor Nodes
variance = [24.025e-9,23.695e-9,25.678e-9,0.0323e-9];
pfa = 0.05;%Probability of false alarm
for i=1:4
threshold(i) = (qfuncinv(pfa)+sqrt(N(i)))*sqrt(N(i))*2*variance(i);
end
% SNR Values from four sensor nodes
snrpractical = [-21.45,-18.23,-15.45,-13.3,-12.67,-9.35,-2.23,-4.32,2.98,6.95,13.57,21.78;...
                -22.23,-20.22,-17.34,-15.32,-13.45,-11.27,-7.75,-5.67,2.53,4.78,12.67,20.32;...
                -25.34,-23.34,-21.67,-20.33,-16.76,-13.38,-8.56,-6.53,0.38,1.34,7.89,16.54;...
                -27.32,-24.97,-22.34,-22.23,-17.34,-14.32,-11.35,-7.85,-2.35,-0.98,2.38,5.98];
for i=1:4
    snrlinearprac(i,:) = 10.^(snrpractical(i,:)/10);
end
for i=1:4
pdprac(i,:) = qfunc((threshold(i)-2*N(i)*variance(i).*(1+snrlinearprac(i,:)))./...
    (sqrt(N(i)*(1+2*snrlinearprac(i,:)))*(2*variance(i))));
end
for i=1:12
snravg(i) = mean(snrpractical(:,i));
end
%% MNE based CS..
pdpracmne = 1-(1-pdprac(1,:)).*(1-pdprac(2,:)).*(1-pdprac(3,:)).*(1-pdprac(4,:));
pdpracand = mean(pdprac).^4;
pdpracm = mean(pdprac);
pdpracor = 1-(1-pdpracm).^4;
tmp1 =  (1-pdpracm).^(k-2);
tmp2 = (1-pdpracm);
pdpracmjr = (6*pdpracm.^(k-2)).*tmp1+(4*pdpracm.^(k-1)).*tmp2+pdpracm.^k;
figure(1)
hold on;
grid on;
plot(snravg,pdpracmne,'-<','LineWidth',2,'MarkerFaceColor','auto');
axis([-20 10 0 1])
%% EGC based CS..
snrlinearmean = 10.^(snravg/10);
snrlinear = 10.^(snrpractical/10);
pfa = 0.01;
M= mean(N);
threshold = mean(threshold);
for i=1:length(snravg)
    num(i) = threshold-snrlinearmean(i);
    den(i) = (1/16)*((1+2*snrlinear(1,i))/N(1)+variance(1)+(1+2*snrlinear(2,i))/N(2)+variance(2)+(1+2*snrlinear(3,i))/N(3)+variance(3)+(1+2*snrlinear(4,i))/N(4)+variance(4));
    pdegc(i) = qfunc(num(i)/sqrt(den(i)));
end

%% Plotting the data..
plot(snravg,pdegc,'-d','LineWidth',2,'MarkerFaceColor','auto');
plot(snravg,pdpracand,'-s','LineWidth',2,'MarkerFaceColor','auto');
plot(snravg,pdpracor,'-s','LineWidth',2,'MarkerFaceColor','auto');
plot(snravg,pdpracmjr,'->','LineWidth',2,'MarkerFaceColor','auto');
title('P_{davg} vs SNR_{avg} for Soft and Hard Decision Combining');
xlabel('SNR_{avg} in (dB)');
ylabel('Average Probability of Detection');
set(gca,'fontsize',30,'box','on','LineWidth',2,'GridLineStyle','--','GridAlpha',0.7);
legend('MNE Combining','EGC Combining','AND Rule','OR Rule','Majority Rule');
\end{lstlisting}

\section{spectrumsenseusrp.py}
\begin{lstlisting}[breaklines]
#!/usr/bin/env python
#
# Copyright 2005,2007,2011 Free Software Foundation, Inc.
#
# This file is part of GNU Radio
#
# GNU Radio is free software; you can redistribute it and/or modify
# it under the terms of the GNU General Public License as published by
# the Free Software Foundation; either version 3, or (at your option)
# any later version.
#
# GNU Radio is distributed in the hope that it will be useful,
# but WITHOUT ANY WARRANTY; without even the implied warranty of
# MERCHANTABILITY or FITNESS FOR A PARTICULAR PURPOSE.  See the
# GNU General Public License for more details.
#
# You should have received a copy of the GNU General Public License
# along with GNU Radio; see the file COPYING.  If not, write to
# the Free Software Foundation, Inc., 51 Franklin Street,
# Boston, MA 02110-1301, USA.
#

from gnuradio import gr, eng_notation
from gnuradio import blocks
from gnuradio import audio
from gnuradio import filter
from gnuradio import fft
from gnuradio import uhd
from gnuradio.eng_option import eng_option
from optparse import OptionParser
import sys
import math
import struct
import threading
from datetime import datetime
import time
from gnuradio.wxgui import stdgui2, fftsink2, form
import wx

sys.stderr.write("Warning: this may have issues on some machines+Python version combinations to seg fault due to the callback in bin_statitics.\n\n")

class ThreadClass(threading.Thread):
    def run(self):
        return

class tune(gr.feval_dd):
    """
    This class allows C++ code to callback into python.
    """
    def __init__(self, tb):
        gr.feval_dd.__init__(self)
        self.tb = tb

    def eval(self, ignore):
        """
        This method is called from blocks.bin_statistics_f when it wants
        to change the center frequency.  This method tunes the front
        end to the new center frequency, and returns the new frequency
        as its result.
        """

        try:
            new_freq = self.tb.set_next_freq()
            while(self.tb.msgq.full_p()):
                time.sleep(0.1)
            return new_freq

        except Exception, e:
            print "tune: Exception: ", e


class parse_msg(object):
    def __init__(self, msg):
        self.center_freq = msg.arg1()
        self.vlen = int(msg.arg2())
        assert(msg.length() == self.vlen * gr.sizeof_float)
        t = msg.to_string()
        self.raw_data = t
        self.data = struct.unpack('%df' % (self.vlen,), t)


class my_top_block(gr.top_block):

    def __init__(self):
        gr.top_block.__init__(self)

        usage = "usage: %prog [options] min_freq max_freq"
        parser = OptionParser(option_class=eng_option, usage=usage)
        parser.add_option("-a", "--args", type="string", default="",
                          help="UHD device device address args [default=%default]")
        parser.add_option("", "--spec", type="string", default=None,
	                  help="Subdevice of UHD device where appropriate")
        parser.add_option("-A", "--antenna", type="string", default=None,
                          help="select Rx Antenna where appropriate")
        parser.add_option("-s", "--samp-rate", type="eng_float", default=1e6,
                          help="set sample rate [default=%default]")
        parser.add_option("-g", "--gain", type="eng_float", default=None,
                          help="set gain in dB (default is midpoint)")
        parser.add_option("", "--tune-delay", type="eng_float",
                          default=0.25, metavar="SECS",
                          help="time to delay (in seconds) after changing frequency [default=%default]")
        parser.add_option("", "--dwell-delay", type="eng_float",
                          default=0.25, metavar="SECS",
                          help="time to dwell (in seconds) at a given frequency [default=%default]")
        parser.add_option("-b", "--channel-bandwidth", type="eng_float",
                          default=6.25e3, metavar="Hz",
                          help="channel bandwidth of fft bins in Hz [default=%default]")
        parser.add_option("-l", "--lo-offset", type="eng_float",
                          default=0, metavar="Hz",
                          help="lo_offset in Hz [default=%default]")
        parser.add_option("-q", "--squelch-threshold", type="eng_float",
                          default=None, metavar="dB",
                          help="squelch threshold in dB [default=%default]")
        parser.add_option("-F", "--fft-size", type="int", default=None,
                          help="specify number of FFT bins [default=samp_rate/channel_bw]")
        parser.add_option("", "--real-time", action="store_true", default=False,
                          help="Attempt to enable real-time scheduling")

        (options, args) = parser.parse_args()
        if len(args) != 2:
            parser.print_help()
            sys.exit(1)

        self.channel_bandwidth = options.channel_bandwidth

        self.min_freq = eng_notation.str_to_num(args[0])
        self.max_freq = eng_notation.str_to_num(args[1])

        if self.min_freq > self.max_freq:
            # swap them
            self.min_freq, self.max_freq = self.max_freq, self.min_freq

        if not options.real_time:
            realtime = False
        else:
            # Attempt to enable realtime scheduling
            r = gr.enable_realtime_scheduling()
            if r == gr.RT_OK:
                realtime = True
            else:
                realtime = False
                print "Note: failed to enable realtime scheduling"

        # build graph
        self.u = uhd.usrp_source(device_addr=options.args,
                                 stream_args=uhd.stream_args('fc32'))

        # Set the subdevice spec
        if(options.spec):
            self.u.set_subdev_spec(options.spec, 0)

        # Set the antenna
        if(options.antenna):
            self.u.set_antenna(options.antenna, 0)

        self.u.set_samp_rate(options.samp_rate)
        self.usrp_rate = usrp_rate = self.u.get_samp_rate()

        self.lo_offset = options.lo_offset

        if options.fft_size is None:
            self.fft_size = int(self.usrp_rate/self.channel_bandwidth)
        else:
            self.fft_size = options.fft_size

        self.squelch_threshold = options.squelch_threshold

        s2v = blocks.stream_to_vector(gr.sizeof_gr_complex, self.fft_size)

        mywindow = filter.window.blackmanharris(self.fft_size)
        ffter = fft.fft_vcc(self.fft_size, True, mywindow, True)
        power = 0
        for tap in mywindow:
            power += tap*tap
        c2mag = blocks.complex_to_mag_squared(self.fft_size)
        self.freq_step = self.nearest_freq((0.75 * self.usrp_rate), self.channel_bandwidth)
        self.min_center_freq = self.min_freq + (self.freq_step/2)
        nsteps = math.ceil((self.max_freq - self.min_freq) / self.freq_step)
        self.max_center_freq = self.min_center_freq + (nsteps * self.freq_step)
        self.next_freq = self.min_center_freq
        tune_delay  = max(0, int(round(options.tune_delay * usrp_rate / self.fft_size)))  # in fft_frames
        dwell_delay = max(1, int(round(options.dwell_delay * usrp_rate / self.fft_size))) # in fft_frames
        self.msgq = gr.msg_queue(1)
        self._tune_callback = tune(self)        # hang on to this to keep it from being GC'd
        stats = blocks.bin_statistics_f(self.fft_size, self.msgq,
                                        self._tune_callback, tune_delay,
                                        dwell_delay)
	self.connect(self.u, s2v, ffter, c2mag, stats)

        if options.gain is None:

            g = self.u.get_gain_range()
            options.gain = float(g.start()+g.stop())/2.0

        self.set_gain(options.gain)
        print "gain =", options.gain

    def set_next_freq(self):
        target_freq = self.next_freq
        self.next_freq = self.next_freq + self.freq_step
        if self.next_freq >= self.max_center_freq:
            self.next_freq = self.min_center_freq

        if not self.set_freq(target_freq):
            print "Failed to set frequency to", target_freq
            sys.exit(1)

        return target_freq


    def set_freq(self, target_freq):
        """
        Set the center frequency we're interested in.

        Args:
            target_freq: frequency in Hz
        @rypte: bool
        """

        r = self.u.set_center_freq(uhd.tune_request(target_freq, rf_freq=(target_freq + self.lo_offset),rf_freq_policy=uhd.tune_request.POLICY_MANUAL))
        if r:
            return True

        return False

    def set_gain(self, gain):
        self.u.set_gain(gain)

    def nearest_freq(self, freq, channel_bandwidth):
        freq = round(freq / channel_bandwidth, 0) * channel_bandwidth
        return freq

def main_loop(tb):

    def bin_freq(i_bin, center_freq):
        freq = center_freq - (tb.usrp_rate / 2) + (tb.channel_bandwidth * i_bin)
        return freq

    bin_start = int(tb.fft_size * ((1 - 0.25) / 2))
    bin_stop = int(tb.fft_size - bin_start)
    fid = open("./usrp.dat","wb")
    while 1:
	m = parse_msg(tb.msgq.delete_head())
        for i_bin in range(bin_start, bin_stop):
            center_freq = m.center_freq
            freq = bin_freq(i_bin, center_freq)
	    power_db = 10*math.log10(m.data[i_bin]/tb.usrp_rate)
	    signal = m.data[i_bin]/(tb.usrp_rate)

            if (power_db > tb.squelch_threshold) and (freq >= tb.min_freq) and (freq <= tb.max_freq):
		print freq,signal,power_db
		fid.write(struct.pack('<f',signal))
    fid.close()#closing the file

if __name__ == '__main__':
    t = ThreadClass()
    t.start()

    tb = my_top_block()
    try:
        tb.start()
        main_loop(tb)

    except KeyboardInterrupt:
        pass
\end{lstlisting}

\section{gnuradiortlsdrsense.py}
\begin{lstlisting}[breaklines]
#!/usr/bin/env python2
# -*- coding: utf-8 -*-
##################################################
# GNU Radio Python Flow Graph
# Title: DTv Spectrum Sensing
# Author: Gill
# Description: Frequency Sweep for UHF White Spaces
# Generated: Fri Mar 10 14:30:20 2017
##################################################

if __name__ == '__main__':
    import ctypes
    import sys
    if sys.platform.startswith('linux'):
        try:
            x11 = ctypes.cdll.LoadLibrary('libX11.so')
            x11.XInitThreads()
        except:
            print "Warning: failed to XInitThreads()"

from PyQt4 import Qt
from gnuradio import blocks
from gnuradio import eng_notation
from gnuradio import fft
from gnuradio import gr
from gnuradio import qtgui
from gnuradio.eng_option import eng_option
from gnuradio.fft import window
from gnuradio.filter import firdes
from optparse import OptionParser
import numpy as np
import osmosdr
import sip
import sys
import time


class spectrum_sensing(gr.top_block, Qt.QWidget):

    def __init__(self):
        gr.top_block.__init__(self, "DTv Spectrum Sensing")
        Qt.QWidget.__init__(self)
        self.setWindowTitle("DTv Spectrum Sensing")
        try:
            self.setWindowIcon(Qt.QIcon.fromTheme('gnuradio-grc'))
        except:
            pass
        self.top_scroll_layout = Qt.QVBoxLayout()
        self.setLayout(self.top_scroll_layout)
        self.top_scroll = Qt.QScrollArea()
        self.top_scroll.setFrameStyle(Qt.QFrame.NoFrame)
        self.top_scroll_layout.addWidget(self.top_scroll)
        self.top_scroll.setWidgetResizable(True)
        self.top_widget = Qt.QWidget()
        self.top_scroll.setWidget(self.top_widget)
        self.top_layout = Qt.QVBoxLayout(self.top_widget)
        self.top_grid_layout = Qt.QGridLayout()
        self.top_layout.addLayout(self.top_grid_layout)

        self.settings = Qt.QSettings("GNU Radio", "spectrum_sensing")
        self.restoreGeometry(self.settings.value("geometry").toByteArray())

        ##################################################
        # Variables
        ##################################################
        self.samp_rate = samp_rate = int(2e6)
        self.freq = freq = 450e6
        self.N = N = 1000

        ##################################################
        # Blocks
        ##################################################
        self.rtlsdr_source_0 = osmosdr.source( args="numchan=" + str(1) + " " + '' )
        self.rtlsdr_source_0.set_time_source('external', 0)
        self.rtlsdr_source_0.set_sample_rate(samp_rate)
        self.rtlsdr_source_0.set_center_freq(freq, 0)
        self.rtlsdr_source_0.set_freq_corr(0, 0)
        self.rtlsdr_source_0.set_dc_offset_mode(2, 0)
        self.rtlsdr_source_0.set_iq_balance_mode(0, 0)
        self.rtlsdr_source_0.set_gain_mode(True, 0)
        self.rtlsdr_source_0.set_gain(15, 0)
        self.rtlsdr_source_0.set_if_gain(15, 0)
        self.rtlsdr_source_0.set_bb_gain(15, 0)
        self.rtlsdr_source_0.set_antenna('', 0)
        self.rtlsdr_source_0.set_bandwidth(0, 0)
          
        self.qtgui_freq_sink_x_0 = qtgui.freq_sink_c(
        	1024, #size
        	firdes.WIN_BLACKMAN_hARRIS, #wintype
        	0, #fc
        	samp_rate, #bw
        	"Recieved Signal", #name
        	1 #number of inputs
        )
        self.qtgui_freq_sink_x_0.set_update_time(0.10)
        self.qtgui_freq_sink_x_0.set_y_axis(-120, 0)
        self.qtgui_freq_sink_x_0.set_y_label('Relative Gain', 'dB')
        self.qtgui_freq_sink_x_0.set_trigger_mode(qtgui.TRIG_MODE_FREE, 0.0, 0, "")
        self.qtgui_freq_sink_x_0.enable_autoscale(True)
        self.qtgui_freq_sink_x_0.enable_grid(True)
        self.qtgui_freq_sink_x_0.set_fft_average(1.0)
        self.qtgui_freq_sink_x_0.enable_axis_labels(True)
        self.qtgui_freq_sink_x_0.enable_control_panel(False)
        
        if not True:
          self.qtgui_freq_sink_x_0.disable_legend()
        
        if "complex" == "float" or "complex" == "msg_float":
          self.qtgui_freq_sink_x_0.set_plot_pos_half(not True)
        
        labels = ['', '', '', '', '',
                  '', '', '', '', '']
        widths = [2, 1, 1, 1, 1,
                  1, 1, 1, 1, 1]
        colors = ["blue", "red", "green", "black", "cyan",
                  "magenta", "yellow", "dark red", "dark green", "dark blue"]
        alphas = [1.0, 1.0, 1.0, 1.0, 1.0,
                  1.0, 1.0, 1.0, 1.0, 1.0]
        for i in xrange(1):
            if len(labels[i]) == 0:
                self.qtgui_freq_sink_x_0.set_line_label(i, "Data {0}".format(i))
            else:
                self.qtgui_freq_sink_x_0.set_line_label(i, labels[i])
            self.qtgui_freq_sink_x_0.set_line_width(i, widths[i])
            self.qtgui_freq_sink_x_0.set_line_color(i, colors[i])
            self.qtgui_freq_sink_x_0.set_line_alpha(i, alphas[i])
        
        self._qtgui_freq_sink_x_0_win = sip.wrapinstance(self.qtgui_freq_sink_x_0.pyqwidget(), Qt.QWidget)
        self.top_layout.addWidget(self._qtgui_freq_sink_x_0_win)
        self.fft_vxx_0 = fft.fft_vcc(1024, True, (window.blackmanharris(1024)), True, 1)
        self.blocks_vector_to_stream_0 = blocks.vector_to_stream(gr.sizeof_float*1, 1024)
        self.blocks_stream_to_vector_0 = blocks.stream_to_vector(gr.sizeof_gr_complex*1, 1024)
        self.blocks_moving_average_xx_0 = blocks.moving_average_ff(N, 1, 4000)
        self.blocks_file_sink_2_0 = blocks.file_sink(gr.sizeof_float*1, '/home/gill/Desktop/ms-thesis/gr-spectrumsensing/grc/rtl-sdr_sensing/Results/snr_check.dat', False)
        self.blocks_file_sink_2_0.set_unbuffered(False)
        self.blocks_complex_to_mag_squared_0 = blocks.complex_to_mag_squared(1024)

        ##################################################
        # Connections
        ##################################################
        self.connect((self.blocks_complex_to_mag_squared_0, 0), (self.blocks_vector_to_stream_0, 0))    
        self.connect((self.blocks_moving_average_xx_0, 0), (self.blocks_file_sink_2_0, 0))    
        self.connect((self.blocks_stream_to_vector_0, 0), (self.fft_vxx_0, 0))    
        self.connect((self.blocks_vector_to_stream_0, 0), (self.blocks_moving_average_xx_0, 0))    
        self.connect((self.fft_vxx_0, 0), (self.blocks_complex_to_mag_squared_0, 0))    
        self.connect((self.rtlsdr_source_0, 0), (self.blocks_stream_to_vector_0, 0))    
        self.connect((self.rtlsdr_source_0, 0), (self.qtgui_freq_sink_x_0, 0))    

    def closeEvent(self, event):
        self.settings = Qt.QSettings("GNU Radio", "spectrum_sensing")
        self.settings.setValue("geometry", self.saveGeometry())
        event.accept()

    def get_samp_rate(self):
        return self.samp_rate

    def set_samp_rate(self, samp_rate):
        self.samp_rate = samp_rate
        self.rtlsdr_source_0.set_sample_rate(self.samp_rate)
        self.qtgui_freq_sink_x_0.set_frequency_range(0, self.samp_rate)

    def get_freq(self):
        return self.freq

    def set_freq(self, freq):
        self.freq = freq
        self.rtlsdr_source_0.set_center_freq(self.freq, 0)

    def get_N(self):
        return self.N

    def set_N(self, N):
        self.N = N
        self.blocks_moving_average_xx_0.set_length_and_scale(self.N, 1)


def main(top_block_cls=spectrum_sensing, options=None):

    from distutils.version import StrictVersion
    if StrictVersion(Qt.qVersion()) >= StrictVersion("4.5.0"):
        style = gr.prefs().get_string('qtgui', 'style', 'raster')
        Qt.QApplication.setGraphicsSystem(style)
    qapp = Qt.QApplication(sys.argv)

    tb = top_block_cls()
    tb.start()
    tb.show()

    def quitting():
        tb.stop()
        tb.wait()
    qapp.connect(qapp, Qt.SIGNAL("aboutToQuit()"), quitting)
    qapp.exec_()


if __name__ == '__main__':
    main()
\end{lstlisting}


