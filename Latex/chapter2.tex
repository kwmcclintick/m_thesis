\chapter{Data Synthesis and Neural Networks}
\label{chapter2}
The differences between physical layer ML and other ML domains are highlighted in~\cite{o2016radio}, drawing attention to the unique challenges the physical layer ML domain faces. Although transmitted wave-forms begin as well defined, man-made, synthetic structures, a virtually endless number of probabalistic, and sometimes non-linear, phenomenom alter the observed wave-forms receive-side. Even within a single phenomenom, there can exist again a virtually endless number of realizations or types of that imperfection. Some of the most prevelent and common imperfections include:
\begin{enumerate}
	\item Carrier Frequency Offset of both the transmitter and receiver's local oscillators, which drive each radio's mixers
	\item Phase Ambiguity introduced by the unknown distance between transmitter and receiver
	\item Random Symbol Timing Offset resulting from independently running sample clocks
	\item Additive White Gaussian Noise, bursty noise from phenomenom such as weather, interference from same and adjacent channels
	\item Wide-band phase rotation resulting in constructive and destructive phase wave-form copies created by reflectors
	\item Narrow-band Delay Spread due to multi-path interference
	\item Random signal arrival times due to MAC and Network layer architectures and schedules
	\item Doppler Shifts resulting from motion of the transmitter, receiver, or scatterers and reflectors within the wireless channel
\end{enumerate}

\section{The Radio Frequency Front End}
tx rx chain, complications, ways to correct

\section{Classical Channel Models}
list the modeling algorithms

\section{Machine Learning Algorithms}
loss functions, etc

\section{Neural Network Architectures}
layers, roles, types, etc


\section{Summary}
review and conclude\\
